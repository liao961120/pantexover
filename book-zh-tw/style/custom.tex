%%%%%%% Packages %%%%%
\usepackage[paper=b5paper, margin=0.9in]{geometry}
\usepackage[heading, fontset = none]{ctex}  % Chinese typesetting
\usepackage{fontspec}           % mainfont
\usepackage{xeCJK}              % Chinese fonts
\usepackage{fancyhdr}           % fancy header
\usepackage[most]{tcolorbox}    % Custom box
\usepackage[explicit]{titlesec}
\usepackage{setspace}           % linestretch
\usepackage{imakeidx}           % index
%%%%%%%%%%%%%%%%%%%%%%


%%%%%%% Font Setup %%%%%%%%%%%%%
% https://www.overleaf.com/learn/latex/Questions/Which_OTF_or_TTF_fonts_are_supported_via_fontspec%3F
\setmainfont[]{Georgia}  % PT Serif
\setsansfont[]{PT Sans}
\setmonofont[]{Ubuntu Mono}
\xeCJKsetup{AutoFallBack, CJKmath=true}
\setCJKmainfont[
    % ItalicFont={AR PL KaitiM Big5},
    % AutoFakeSlant=0.1
]{Noto Serif CJK TC}
\setCJKsansfont{Noto Sans CJK TC}
\setCJKmonofont[]{Noto Sans Mono CJK TC}
% Custom fonts
% \setCJKmainfont[Path=fonts/,AutoFakeBold=2.5,AutoFakeSlant=.3]{kaiti}
%%%%%%%%%%%%%%%%%%%%%%%%%%%%%%%%%%%%%%%%%%%%


%%%%%%%%%%%%%%% 中文標題格式 %%%%%%%%%%%%%%%%%%%%%%
%% Appendix toc alignment
% https://tex.stackexchange.com/questions/7853/toc-text-numbers-alignment
% \usepackage{tocloft}
% \setlength{\cftchapnumwidth}{3em}
\ctexset{
    part/name = {第, 部分},
    chapter/name = {第,章},
    chapter/aftername = {
        \\
    },
    % chapter/aftertitle = {
    %      \vspace*{-2.19ex}\noindent\rule{\textwidth}{0.4pt}
    % },
    chapter/format = \Huge\bfseries\raggedright,
    chapter/nameformat = \LARGE\bfseries\raggedright,
    chapter/numberformat = \sffamily,
    section/numberformat = \sffamily,
    subsection/numberformat = \sffamily,
    subsubsection/numberformat = \sffamily,
    section/format = \Large\bfseries\raggedright,
    chapter/number = \arabic{chapter},
    % subsection/format += \fbox,
    appendix/name={\appendixname\space}
}
%%%%%%%%%%%%%%%%%%%%%%%%%%%%%%%%%%%%%%%%%%%%%%%%%%%%%%%



%%%%%%%%%%% Title names %%%%%%%%%%%%%%%%
\renewcommand{\chaptername}{\CTEXthechapter}
\renewcommand{\contentsname}{目錄}
\renewcommand{\figurename}{圖}
\renewcommand{\tablename}{表}
\renewcommand{\listfigurename}{圖目錄}
\renewcommand{\listtablename}{表目錄}
\renewcommand{\appendixname}{附錄}
%%%%%%%%%%%%%%%%%%%%%%%%%%%%%%%%%%%%%%%%


%%%%%%%%%%%%%%% Fancy header %%%%%%%%%%%%%%%%%%
\makeatletter
    \fancyhf{}
    \fancyhead[EL]{\nouppercase\leftmark}
    \fancyhead[OR]{\nouppercase\rightmark}
    \fancyhead[ER,OL]{\thepage}
    \pagestyle{fancy}
    %% Patch for blank pages between chapters
    %% which disables the current page style
    \def\cleardoublepage{\clearpage\if@twoside \ifodd\c@page\else
      \hbox{}\thispagestyle{empty}\newpage\if@twocolumn\hbox{}\newpage\fi\fi\fi}
\makeatother
%%%%%%%%%%%%%%%%%%%%%%%%%%%%%%%%%%%%%%%%%%%%%%%


%%%%%% Line numbering for all highlighted code chunks %%%%%
% Overrides pandoc code chunk setup
% ori: \DefineVerbatimEnvironment{Highlighting}{Verbatim}{commandchars=\\\{\}}
\RecustomVerbatimEnvironment{Highlighting}{Verbatim}{
    commandchars=\\\{\},
    numbers=left,
    frame=leftline,  % single, lines, leftline, topline
    % xleftmargin=5.3mm,
    numbersep=7pt,
    % framerule=.4mm,
    rulecolor=\color{gray}
    }
%%%%%%%%%%%%%%%%%%%%%%%%%%%%%%%%%%%%%%%%%%%%%%%%%%%%%%%%%%


%%%%%%% Globally change verbatim font size %%%%%%%%%%
% \usepackage{etoolbox}
\makeatletter
\patchcmd{\@verbatim}
  {\verbatim@font}
  {\verbatim@font\small}
  {}{}
\makeatother
%%%%%%%%%%%%%%%%%%%%%%%%%%%


%%%%%%%% Custom Box %%%%%%%%%%%%%%%%%%%%%%%
\newenvironment{mybox}[1]
    {   
        \vspace{12pt}
        \begin{tcolorbox}[
            enhanced, breakable,
            colback=red!5!white, 
            colframe=red!50!black, 
            title=#1] }
    {  \end{tcolorbox} }
%%%%%%%%%%%%%%%%%%%%%%%%%%%%%%%%%%%%%%%%%%


%%%%%%%% Linestretch (Doesn't change footnotes) %%%%%%%%%%%%%%
\setstretch{1.43}
% Set independent linestretch for code & captions
\let\oldShaded=\Shaded
\let\endoldShaded=\endShaded
\renewenvironment{Shaded}{
    \begin{spacing}{1}\begin{oldShaded}
  }
  {
  \end{oldShaded}
  \end{spacing}
  \vspace{0.1cm}
  }
\let\oldLongTable=\longtable
  \let\endoldLongTable=\endlongtable
  \renewenvironment{longtable}{
      \begin{small}\begin{spacing}{1.1}\begin{oldLongTable}
    }
    {
    \end{oldLongTable}\end{spacing}\end{small}
    }
\let\oldTable=\table
  \let\endoldTable=\endtable
  \renewenvironment{table}{
    \begin{small}\begin{spacing}{1.1}\begin{oldTable}
    }
    {
    \end{oldTable}\end{spacing}\end{small}
    }
%%%%%%%%%%%%%%%%%%%%%%%%%%%%%%%%%%%%%%%%%%%%%%%%%%%%%%%%%%%%%%%


%%%%%%%% Custom Section Styling %%%%%%
% src: https://tex.stackexchange.com/questions/599059/fancy-chapter-style-custom
\titleformat{\section}[display]
  {\sffamily\large}
  {}
  {0pt}
  {\hrule\vspace*{0.25ex}\parbox[b]{\dimexpr\textwidth\relax}{
  \textcolor{darkgray}{\textbf{\thesection}}\quad\raggedright\bfseries#1}}
  [\hrule]
%%%%%%%%%%%%%%%%%%%%%%%%%%%%%%%%%%%%%%


%%%%%%%%%% Index %%%%%%%%%%%%%%%%
\makeindex[title=索引, intoc]
%%%%%%%%%%%%%%%%%%%%%%%%%%%%%%%%%